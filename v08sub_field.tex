% field measurements
\subsection{Feldversuche}


\begin{frame}
\frametitle{Vorbetrachtung}
In Feldversuchen werden meist Wellengeschwindigkeiten 
\begin{align*}
 c_\mathrm{P} &= \sqrt{\frac{E_\mathrm{b}}{\rho}} 
 = \sqrt{\frac{G}{\rho}}\sqrt{\frac{2(1-\nu)}{1-2\nu}} \\
 c_\mathrm{S} &= \sqrt{\frac{G}{\rho}} 
\end{align*}
gemessen, aus denen $G$ und $E_b$ unmittelbar folgen.
Die Querdehnung ergibt sich aus
\begin{equation*}
\nu = \frac{c_\mathrm{P}^2 - 2c_\mathrm{S}^2}{2(c_\mathrm{P}^2-c_\mathrm{S}^2)}  .
\end{equation*}
\end{frame}


\begin{frame}
\frametitle{Refraktionsseismik}
siehe Wellen, Transmission

\end{frame}


\begin{frame}
\frametitle{Bohrlochmessungen}

Cross-hole, Up-hole, Down-hole
\end{frame}


\begin{frame}
\frametitle{R-Wellen Dispersionsmessung}
0.3 lambda

\end{frame}
