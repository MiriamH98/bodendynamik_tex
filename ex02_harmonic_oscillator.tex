\question{Erzwungene, ungedämpfte Schwingung}
\vspace{1em}

\begin{minipage}[t]{.49\linewidth}
geg.:
\begin{tasks}(2)
    \task[] $m = \SI{3}{\kilo\gram}$
    \task[] $k = \SI{10}{\newton\per\meter}$
    \task[] $F(t) = \hat{F} \sin{\omega t}$
    \task[] $\hat{F} = \SI{5}{\newton}$
    \task[] $\omega = \SI{10}{\radian\per\second}$
    \task[] $t=\SI{0}{\second}$
    \task[] $u_0 = \SI{0.1}{m}$
    \task[] $\dot{u}_0 = \SI{0}{\meter\per\second}$
\end{tasks}
\end{minipage}
\begin{minipage}[t]{.49\linewidth}
ges.:
\begin{tasks}
    \task $u(t)$
\end{tasks}
\end{minipage}

\begin{solution}
    \begin{alignat*}{2}
        &(c = \SI{0}{\newton\second\per\meter}, F_C = \SI{0}{\newton}, F_S = \hat{F} = \SI{5}{\newton})\\
        &\omega_0 = \sqrt{k/m} &&= \SI{1.826}{\radian\per\second} \\
        &\zeta = \frac{c}{2m\sqrt{k/m}} &&= 0 \\
        &\eta = \omega/\omega_0 &&= 5.477\\
        &V = (1-\eta^2)^{-1} &&=  0.0345 \\
        &u_C = -2V^2\zeta\eta\hat{F}/k &&= \SI{0}{\meter} \\
        &u_S = V^2(1-\eta^2)\hat{F}/k &&= \SI{-0.0172}{\meter}
    \end{alignat*}

    \begin{align*}
        u_0 &= C_1 + u_C \\
        \dot{u}_0 &= \omega_0 C_2+\omega u_S\\
        &\rightsquigarrow \\
        C_1 &=  u_0 - U_C = \SI{0.1}{\meter}\\
        C_2 &= \frac{\dot{u}_0 - \omega u_S}{\omega_0} = \SI{0.0944}{\meter} \\
        \vspace{1cm}
    \end{align*}
    \begin{equation*}
        \begin{split}
            u(t) = \SI{0.1}{\meter} \cos{(\SI{1.826}{\radian\per\second} t)} +\SI{0.0944}{\meter} \sin{((\SI{1.826}{\radian\per\second} t))} \\ - \SI{0.0172}{\meter}\sin{(\SI{10}{\radian\per\second} t)}
        \end{split}
    \end{equation*}
        \fbox{\textbf{Anmerkung:} Das Einschwingen dauert im  ungedämpften Fall unendlich lang.}
\end{solution}
    
    
%%%%%%%%% Übungsaufgabe 2 %%%%%%%%%

\question{Erzwungene, gedämpfte Schwingung}

    \begin{minipage}[t]{.49\linewidth}
        geg.:
        \begin{tasks} (2)
           \task[] $m = \SI{3}{\kilo\gram}$
           \task[] $c = \SI{1}{\newton\second\per\meter}$
           \task[] $k = \SI{10}{\newton\per\meter}$
            \task[] $F(t) = \hat{F} \sin{\omega t}$
           \task[] $\hat{F} = \SI{5}{\newton}$
           \task[] $\omega = \SI{10}{\radian\per\second}$
           \task[] $t=\SI{0}{\second}$
           \task[] $u_0 = \SI{0.1}{m}$
           \task[] $\dot{u}_0 = \SI{0}{\meter\per\second}$
           \task[] $t_0 = \SI{0}{\second}$
        \end{tasks}
        \end{minipage}
        \begin{minipage}[t]{.49\linewidth}
        ges.:
        \begin{tasks}
            \task[] $u(t)$
        \end{tasks}
    \end{minipage}\\
    \vspace{1cm}

    \underline{Zusatzfrage:} Wie ändert sich die Lösung, wenn die Anregung zusätzlich einen Konstantanteil enthält $F(t) = F_{DC} +\hat{F} \sin{(\omega t)}$ (Stichwort \textit{statische Ruhelage})?

    \begin{solution}
        \begin{alignat*}{2}
            &\omega_0 = \sqrt{k/m} &&= \SI{1.826}{\radian\per\second} \\
            &\zeta = \frac{c}{2m\sqrt{k/m}} &&= 0.029 \text{Bitte nochmal berechnen!} \\
            &\eta = \omega/\omega_0 &&= 5.477\\
            &V = \frac{1}{\sqrt{(1 - \eta^2)^2 + (2\zeta\eta)^2}} &&= 0.034\\
            &\text{da: } a(t) = \frac{F(t)}{k} \sin(\omega t) = \hat{a} \cos(\psi_a) \cos(\omega t) + \hat{a} \sin{\psi_a} \sin{\omega t} \\
            &u_c = V^2((1-\eta^2)a_c - 2 \zeta \eta a_s) \\
            &u_s = V^2((1 - \eta^2)a_s + 2 \zeta \eta a_c) \\
        \end{alignat*}
    \end{solution}
        
\begin{figure}[h]
    \centering
    \begin{gnuplot}[terminal=epslatex, terminaloptions={size 15cm,5cm}]
       set zeroaxis
       unset border
       set xtics axis
       unset ytics
       unset key

       set xr [0:3*pi]
       set yr [-1:1]
       set xl "t" offset 27,7,0
       set yl "u" offset 2,6,0 rotate by 0

       set xtics ("$t_{max1}$" pi/4, "$t_{max2}$" pi/4+2*pi)

       set label 1 "" at pi/4,1 point pointtype 7 pointsize 2 lc 0
       set label 2 "" at pi/4+2*pi,1 point pointtype 7 pointsize 2 lc 0

       set arrow 1 from 0,0 to graph 0, first 1 filled head
       set arrow 2 from 0,0 to first 0, graph 0 filled head
       set arrow 3 from 0,0 to graph 1,.5 filled head

       plot cos(x-pi/4) w l lc 0
    \end{gnuplot}
\end{figure}


\question{Ausschwingversuch}

     \begin{minipage}[t]{.49\linewidth}
        geg.:
        \begin{tasks}(1)
           \task[] $m = \SI{3}{\kilo\gram}$
           \task[] $u(t_{max1}) = \SI{0.10}{\meter}$
           \task[] $u(t_{max2}) = \SI{0.08}{\meter}$
           \task[] $t_{max2}-t_{max1} = 1$
        \end{tasks}
        \end{minipage}
        \begin{minipage}[t]{.49\linewidth}
        ges.:
        \begin{tasks}
            \task $c ~,~ k$
        \end{tasks}
        \end{minipage}\\
        \vspace{1cm}
        
        \fbox{\underline{Hinweis:} Nutzen Sie das logarithmische Dekrement $\Lambda = \log \frac{u(t_{max1})}{u(t_{max2})}$ als Zwischenergebnis.}

        \begin{solution}
        HIER BITTE NOCH LÖSUNG NACHTRAGEN\vfill
        \end{solution}

     %%%%%%%%% Übungsaufgabe 4 %%%%%%%%%

 \question{Schwingsaitenwaage}



\begin{minipage}[t]{.49\linewidth}
    geg.:
    \begin{tasks}(1)
        \task[] $m = \SI{3}{\kilo\gram}$
        \task[] $\omega_0 = \SI{10}{\radian\per\second}$
        \task[] $0 < \zeta < 1$
    \end{tasks}
    \end{minipage}
    \begin{minipage}[t]{.49\linewidth}
    ges.:
    \begin{tasks}
        \task $k$
    \end{tasks}
\end{minipage}\\
    \vspace{1cm}
    \underline{Zusatzfrage:} Je nach Messprinzip, wird entweder die ungedämpfte
Eigenfrequenz, die gedämpfte Eigenfrequenz oder die maximale Anwortamplitude
hervorrufende Anregungsfrequenz direkt erfasst. Wie groß ist für $\zeta = 0.1$ der
Fehler bei der Schätzung von $k$, wenn man diese Frequenzen gleichsetzt?


