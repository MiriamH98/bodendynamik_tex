\begin{questions}
    \question{Dämpfungskapazität des äquivalent-linearen Modells}
    \vspace{1em}

        \begin{minipage}[t]{.49\linewidth}
        geg.:
            \begin{tasks} (1)
                \task[] $F_D = 2\, \zeta\, \eta\, k\, \hat u\, \sin(a\, t)$
                \task[] $\dot{u} = \omega\,\hat u\, \sin(a\,t)$
            \end{tasks}
        \end{minipage}
        \begin{minipage}[t]{.49\linewidth}
        ges.:
            \begin{tasks}
                \task[] dissipierte Energie $\Delta W$ pro Zyklus
                \task[] maximal in Feder gespeicherte Energie $W$
                \task[] Dämpfungskapazität $\Psi$
            \end{tasks}
        \end{minipage}
    \vspace{1cm}

    \begin{solution}
        \begin{align*}
            W_D &= \int_{0}^{T} -F_D \dot{u} \dd{t}\\
            \intertext{für $\phi = [0, 2\pi]$} \\
            W_D &= \int_{0}^{2 \pi} F_D \dot{u} \dd{t} &&= 2 \pi \hat{u}^2 \eta k \zeta\\
            \intertext{Maximale, in Feder gespeicherte Energie:} \\
            W_{F_{\max}} &= 0.5 k \hat{u}^2 \\
            \intertext{Dämpfungskapazität:} \\
            \psi &= \frac{W_D}{W_{F_{\max}}} &&= 4.0 \pi \eta \zeta \\
        \end{align*}
    \end{solution}


\question{Schubmodul und Dämpfung bei mittleren Dehnungen}
    \vspace{1em}

    \begin{minipage}[t]{.49\linewidth}
    geg.:
        \begin{tasks} (1)
            \task[] $\hat\gamma = 0,5\%$
            \task[] $I_p = 10 \%$
        \end{tasks}
    \end{minipage}
    \begin{minipage}[t]{.49\linewidth}
    ges.:
        \begin{tasks}
            \task[] $\cfrac{G}{G_{\max}}$
            \task[] $D$
        \end{tasks}
    \end{minipage}

    \begin{center}\fbox{\underline{Hinweis:} siehe Kap. 4.2.4 Grundbau-Taschenbuch}\end{center}

    \begin{solution}
        \begin{align*}
            \frac{G}{G_{max}} 
            & = \frac{1.03}{1+\frac{19 \gamma^{0.8}}{1+ \frac{I_P}{15.0}^{1.3}}} \leq 1  
            & & \gamma \leq 1 \% \\
            D 
            & = 2 + \frac{24.5 - 0.2 I_P}{1 + \frac{1}{7.4 - 0.1 I_P} \gamma^{0.8}} \leq 1 
            & & \gamma \leq 1 \%, I_P \leq 50 \% 
            \end{align*}
            mit den gegebenen Werten
            \begin{align*}
            \frac{G}{G_{max}} &= 0.809,\\
            D &= 26.432.
        \end{align*}
    \end{solution}
\vspace{1cm}

\question{Schubmodul bei sehr kleinen Dehnungen}
\vspace{1em}

     \begin{minipage}[t]{.49\linewidth}
        geg.:
        \begin{tasks} (2)
            \task[] $n = 0.5$
            \task[] $e = 2.0$
            \task[] $I_p = 0.6$
            \task[] $P_a = 100 kPa$
            \task[] $\bar \sigma' = 1 MPa$
            \task[] $OCR = 5$
            \task[] $S = 625$
        \end{tasks}
    \end{minipage}
    \begin{minipage}[t]{.49\linewidth}
    ges.:
        \begin{tasks}
            \task[] $- G_{\max|NC}$
            \task[] $- G_{\max|OC}$
        \end{tasks}
    \end{minipage}
    \begin{center}\fbox{\underline{Hinweis:} siehe Kap. 4.2.2 Grundbau-Taschenbuch}\end{center}

    \begin{solution}
        \begin{align*}
            F_e &= \frac{1}{0.3 + 0.7 e^2} &&= 0.323\\
            G_{\max | NC} &= S\ F_e\ P_a\ (\frac{\bar \sigma'}{P_a})^n &&= \SI{63.76}{\mega \pascal} \\
            k &= \frac{I_P}{(1+ 3 I_P^2)^{0.5}} &&= 0.416 \\
            G_{\max | OC} &= G_{\max | NC}\cdot OCR^k &&= \SI{124.54}{\mega \pascal}\\
        \end{align*}
    \end{solution}

\question{Masing-Hypothese, geometrische Konstruktion}
\vspace{1em}
    \begin{tasks}
        \task[] Zeichnen sie die Skelettkurve (punktweise Berechnung).
        \task[] Konstruieren sie für $\cfrac{\hat \gamma}{\gamma_r}=2$ (Faktor 2) die Hysterese-Äste.
    \end{tasks}

    \begin{align*}
        \cfrac{\tau}{\tau_m} = \cfrac{\cfrac{\gamma}{\gamma_r}}{1+|\cfrac{\gamma}{\gamma_r}|}
    \end{align*}

    %\includegraphics[width=0.99\textwidth]{fig_gnuplottex/graph_template.png}
    graph\_template


    \begin{solution}

        \begin{tikzpicture}
    \begin{axis}[
        title = {Skelettkurve},
        xlabel = {$\gamma / \gamma_r$},
        ylabel = {$\theta / \theta_m$},
        xmin = -20, xmax = 20,
        ymin = -1.5, ymax = 1.5,
        xtick distance = 5,
        ytick distance = 0.25,
        grid = both,
        major grid style = {lightgray},
        minor grid style = {lightgray!25},
        width = 10cm,
        height = 6 cm,
        axis x line = center,
        axis y line = center,
        ]
    \addplot[mark=none, semithick, black, dashed]{\frac{x}{1 + \left\lvert x \right\rvert}};
    \end{axis}
\end{tikzpicture}

        %width = 0.5 \textwidth

    \end{solution}
    
\end{questions}
