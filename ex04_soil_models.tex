\question{Dämpfungskapazität des äquivalent-linearen Modells}
 \vspace{1em}

    \begin{minipage}[t]{.49\linewidth}
    geg.:
    \begin{tasks}(1)
      \task[] $F_D = 2\, \zeta\, \eta\, k\, \hat u\, \sin(a\, t)$
      \task[] $\dot{u} = \omega\,\hat\, u\, \sin(a\,t)$
    \end{tasks}
    \end{minipage}
    \begin{minipage}[t]{.49\linewidth}
    ges.:
        \begin{tasks}
            \task dissipierte Energie $\Delta W$ pro Zyklus
            \task maximal in Feder gespeicherte Energie $W$
            \task Dämpfungskapazität $\Psi$
        \end{tasks}
    \end{minipage}
\vspace{1cm}

 \question{Schubmodul und Dämpfung bei mittleren Dehnungen}
  \vspace{1em}

     \begin{minipage}[t]{.49\linewidth}
    geg.:
    \begin{tasks}(1)
      \task[] $\hat\gamma = 0,5\%$
      \task[] $I_p = 10 \%$
    \end{tasks}
    \end{minipage}
    \begin{minipage}[t]{.49\linewidth}
    ges.:
        \begin{tasks}
            \task $\cfrac{G}{G_{max}}$
            \task $D$
        \end{tasks}
    \end{minipage}

    \begin{center}\fbox{\underline{Hinweis:} siehe Kap. 4.2.4 Grundbau-Taschenbuch}\end{center}

\vspace{1cm}

 \question{Schubmodul bei sehr kleinen Dehnungen}
  \vspace{1em}

     \begin{minipage}[t]{.49\linewidth}
    geg.:
    \begin{tasks}(2)
      \task[] $e = 2,0$
    \task[] $I_p = 0,6$
    \task[] $P_a = 100 kPa$
    \task[] $\bar \sigma' = 1 MPa$
    \task[] $OCR = 5$
    \end{tasks}
    \end{minipage}
    \begin{minipage}[t]{.49\linewidth}
    ges.:
        \begin{tasks}
            \task $- G_{max|NC}$
            \task $- G_{max|OC}$
        \end{tasks}
    \end{minipage}
 \begin{center}\fbox{\underline{Hinweis:} siehe Kap. 4.2.2 Grundbau-Taschenbuch}\end{center}

 \question{Masing-Hypothese, geometrische Konstruktion}
 \vspace{1em}
 \begin{tasks}
 \task Zeichnen sie die Skelettkurve (punktweise Berechnung).
\task Konstruieren sie für $\cfrac{\hat \gamma}{\gamma_r}=2$ (Faktor 2) die Hysterese-Äste.
 \end{tasks}
 \begin{align*}
 \cfrac{\tau}{\tau_m} = \cfrac{\cfrac{\gamma}{\gamma_r}}{1+|\cfrac{\gamma}{\gamma_r}|}
 \end{align*}
    \begin{figure}[!h]
        \centering
        \begin{gnuplot}[terminal=epslatex, terminaloptions={size 15cm,10cm}]
    set xr [-10:10]
    set yr [-1:1]
    set xl '$\cfrac{\cfrac{\gamma}{\gamma_r}}{1+|\cfrac{\gamma}{\gamma_r}|}$' offset 0,-2
    set yl '$\frac{\tau}{\tau_r}$'
    unset key
    set grid
    set zeroaxis
    set title 'Skelettkurve'
    plot x/((1+abs(x)))
\end{gnuplot}

        \label{fig:Skelettkurve}
        \vfill
    \end{figure}

    \newpage

    \begin{figure}[!ht]
    \centering
        \begin{minipage}[t]{\textwidth}
        \centering
            \begin{gnuplot}[terminal=epslatex,terminaloptions={size 15cm,18cm}]
    set key off
    set zeroaxis
    set grid
    set yrange [-1:1]
    set multiplot layout 2,1 rowsfirst

    set title 'zentrische Streckung der Skelettkurve'
    plot 1/0

    set title 'Spiegelung am Koordinatenursprung'
    plot 1/0

    unset multiplot
\end{gnuplot}

        \end{minipage}
    \end{figure}
