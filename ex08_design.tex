Erdbebenersatzkraft anhand des Bemessungsspektrums

\smallskip

TODO Aufgabe V09aufgabe.pdf (handschriftlich)


\question{Erdbebenbemessungspektren}

Ermitteln Sie für ein zweigeschossiges Gastronomiegebäude in Holzbauweise die Gesamterdbebenkraft in horizontaler Richtung
aus dem Bemessungsspektrum.

\begin{minipage}[t]{\linewidth}
    geg:
    \begin{tasks} (2)
        \task[] Erdbebenzone 3
        \task[] $a_{gR} = \SI{0.8}{\metre\per\second^2}$
        \task[] Bedeutungskategorie \romannum{3}
        \task[] $\gamma = 1.2$
        \task[] Untergrundklasse B-R
        \task[]
        \task[] Duktilitätsklasse 1
        \task[] $q = 1.5$
        \task[] Masse: $m =\SI{80}{\tonne}$
        \task[] Gesamthöhe: $h = 6m$
    \end{tasks}
\end{minipage}

Formeln und Parameter aus [Schmidt, Buchmaier, Vogt-Beyer: Grundlagen der Geotechnik, 5. Auflage, S.725f.]

\begin{align*}
    &0 < T \leqq T_B: &&S_d(T) = a_{gR} \gamma_l S [1 + \frac{T}{T_B}(\frac{2.5}{q} -1)] \\
    &T_B \leqq T \leqq T_C: &&S_d(T) = a_{gR} \gamma_l S \frac{2.5}{q} \\
    &T_C \leqq T \leqq T_D: &&S_d(T) = a_{gR} \gamma_l S \frac{2.5}{q} \frac{T_C}{T} \\
    &T_D \leqq T \leqq 4s: &&S_d(T) = a_{gR} \gamma_l S \frac{2.5}{q} \frac{T_C T_D}{T^2} \\
\end{align*}

Anhand der Untergrundklasse B-R lassen sich folgende Parameter ablesen:
\begin{tasks} (4)
    \task[] $S = 1.25$
    \task[] $T_B in s = 0.05$
    \task[] $T_C in s = 0.25$
    \task[] $T_D in s = 2.0$
\end{tasks}

Nach \textit{DIN EN 1198-1} lässt sich die Gesamterdbebenkraft $F_b$ bestimmen.

\begin{align}
    F_b &= S_d(T_1) m \lambda
\end{align}

$T_1$ wird näherungsweise bestimmt durch:
\begin{align}
    T_1 &= C_t h^{\frac{3}{4}}
\end{align}

\begin{solution}
    
\end{solution}