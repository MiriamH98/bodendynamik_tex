\documentclass[hyperref={pdfpagemode=FullScreen, colorlinks=false}]{beamer}

\usepackage{selinput}			% Inputencoding
	\SelectInputMappings{adieresis={ä}, germandbls={ß}, Euro={€}}
\usepackage[T1]{fontenc}		% Fontencoding
%
\usepackage{pifont}
\usepackage{csquotes,siunitx}			% Anführungszeichen; wird von biblatex gewünscht
\usepackage[backend=biber,citestyle=alphabetic,uniquelist=false]{biblatex}	% Literatur formatieren
\addbibresource{bodendynamik.bib}	% Literaturdatenbank
\usepackage{caption} 
\usepackage{subfig}
\usepackage{comment}
%%%%%%%%%%%%%%%%%%%%%%%%%%%%%%%%%%%%%%%%%%%%%%%%%%%%%%%%%%%%%%%%%%%%%%%%%%%%%%%%%%%%%%%%%%%%%%%%%%%%%%%
% Thema für Präsentation
\usetheme[fusszeile=ernstcolor,sprache=ngerman,seite=letzte,
verhaeltnis=16:10,
hausschrift=false,
navigation=false,
titelseite=blau]{TUBAF}

\TUBAFZweitlogo{\includegraphics{fig_pdf/UFZ_logo_inv.pdf}}

%%%%%%%%%%%%%%%%%%%%%%%%%%%%%%%%%%%%%%%%%%%%%%%%%%%%%%%%%%%%%%%%%%%%%%%%%%%%%%%%%%%%%%%%%%%%%%%%%%%%%%%
% Optionen für Anmerkungen
\mode<presentation>{%
\setbeameroption{hide notes}				% keine Notizen (default)
%\setbeameroption{show notes}				% Notizen und Frames gemischt
%\setbeameroption{show only notes}			% nur Notizen
%
%\usepackage{pgfpages}					% wird für nachfolgendes benötigt
%\setbeameroption{show notes on second screen=left}	% wie gesagt; left, right, bottom, top
}




%%% DK packages and settings
\usepackage{amsmath}
\usepackage{pgfpages}
\pgfpagesuselayout{resize to}[a4paper, landscape]   % border shrink=5mm
\usepackage{siunitx}  
%\sisetup{locale = DE} 
\usepackage{tikz}
\usepackage{pgfplots}
\usepackage{animate}

\usetikzlibrary{math}
%\usetikzlibrary{datavisualization.formats.functions}
%\usetikzlibrary{datavisualization}
\usetikzlibrary{intersections}
\usepgfplotslibrary{groupplots,dateplot}
\pgfplotsset{compat=1.16}

\tikzset{
%DKspring(length) length=2...10
DKspring/.pic={
\coordinate (half_up) at (0.5*0.125*#1-0.5*0.125*2, 0.5*0.125*10-0.5*0.125*#1); %at (0.5*(#1-0.2), 0.5*(1.0-#1));
\coordinate (full_up)   at ( 0.125*#1-    0.125*2,     0.125*10-    0.125*#1);
\coordinate (full_down) at ( 0.125*#1-    0.125*2,    -0.125*10+    0.125*#1);
\draw (0, 0) -- ++(1, 0) -- ++(half_up)
    -- ++(full_down) -- ++(full_up) 
    -- ++(full_down) -- ++(full_up)
    -- ++(full_down) -- ++(full_up)
    -- ++(full_down) -- ++(half_up)
    -- ++(1, 0);
    },   
%DKdashpot(length) length=02...10    
DKdashpot/.pic={
\coordinate (upper_end) at (#1-0.5, 0.5);
\coordinate (lower_end) at (#1-0.5,-0.5);
\coordinate (upper_pos) at (#1-1, 0.5);
\coordinate (lower_pos) at (#1-1,-0.5);
\coordinate (center_pos) at (#1-1, 0.0);
\coordinate (center_end) at (#1, 0.0);
\draw (0, 0) -- ++(1, 0);
\draw (upper_end) -- (1, 0.5) -- (1, -0.5) -- (lower_end);
\draw (center_pos) -- (center_end);
\draw (upper_pos) -- (lower_pos);
    },
DKbase/.pic={
\draw[thick] (0, 1.5) -- (0, -1.5);
\foreach \y in {-1.5,-1.0,...,1.0} \draw[thin] (0, \y) -- +(-0.5, 0.5);
},
 invisible/.style={opacity=0},
  visible on/.style={alt={#1{}{invisible}}},
  alt/.code args={<#1>#2#3}{%
    \alt<#1>{\pgfkeysalso{#2}}{\pgfkeysalso{#3}} % \pgfkeysalso doesn't change the path
  }
}
\newlength\figH     % to scale tikzplotlib figures
\newlength\figW     % to scale tikzplotlib figures


\setbeamercovered{transparent}
%-----------------Custom footnote---------------
\TUBAFFzstrikttext{D. Kern \TUBAFfztrenner T. Nagel --- Vorlesung Bodendynamik --- Sommersemester 2021 }
%-----------------------------------------------

\tikzset{
%DKspring(length) length=2...10
DKspring/.pic={
\coordinate (half_up) at (0.5*0.125*#1-0.5*0.125*2, 0.5*0.125*10-0.5*0.125*#1); %at (0.5*(#1-0.2), 0.5*(1.0-#1));
\coordinate (full_up)   at ( 0.125*#1-    0.125*2,     0.125*10-    0.125*#1);
\coordinate (full_down) at ( 0.125*#1-    0.125*2,    -0.125*10+    0.125*#1);
\draw (0, 0) -- ++(1, 0) -- ++(half_up)
    -- ++(full_down) -- ++(full_up) 
    -- ++(full_down) -- ++(full_up)
    -- ++(full_down) -- ++(full_up)
    -- ++(full_down) -- ++(half_up)
    -- ++(1, 0);
    },   
%DKdashpot(length) length=02...10    
DKdashpot/.pic={
\coordinate (upper_end) at (#1-0.5, 0.5);
\coordinate (lower_end) at (#1-0.5,-0.5);
\coordinate (upper_pos) at (#1-1, 0.5);
\coordinate (lower_pos) at (#1-1,-0.5);
\coordinate (center_pos) at (#1-1, 0.0);
\coordinate (center_end) at (#1, 0.0);
\draw (0, 0) -- ++(1, 0);
\draw (upper_end) -- (1, 0.5) -- (1, -0.5) -- (lower_end);
\draw (center_pos) -- (center_end);
\draw (upper_pos) -- (lower_pos);
    },
DKbase/.pic={
\draw[thick] (0, 1.5) -- (0, -1.5);
\foreach \y in {-1.5,-1.0,...,1.0} \draw[thin] (0, \y) -- +(-0.5, 0.5);
},
 invisible/.style={opacity=0},
  visible on/.style={alt={#1{}{invisible}}},
  alt/.code args={<#1>#2#3}{%
    \alt<#1>{\pgfkeysalso{#2}}{\pgfkeysalso{#3}} % \pgfkeysalso doesn't change the path
  }
}


%%%%%%%%%%%%%%%%%%%%%%%%%%%%%%%%%%%%%%%%%%%%%%%%%%%%%%%%%%%%%%%%%%%%%%%%%%%%%%%%%%%%%%%%%%%%%%%%%%%%%%%
% Daten für die Titelseite:
%
% WICHTIG:	german shortcuts funktionieren nicht!! -> ÄäÖöÜüß verwenden
%		\\ fnkt nur im PM, \newline in AM und PM
%
\TUBAFTitel{Bodendynamik}

\TUBAFUntertitel{Dominik Kern, Thomas Nagel}

\TUBAFAutor[D. Kern | T. Nagel]{Dominik Kern, Thomas Nagel}

\TUBAFDatum[SS21]{Sommersemester 2021}

\TUBAFOrt[IFGT/BOME]{Institut für Geotechnik/Lehrstuhl fuer Bodenmechanik und Grundbau}

\TUBAFTitelseiteerlaeuterung{Lehrstuhl Bodenmechanik \& Grundbau\\Institut für Geotechnik\\[0.5cm]Vorlesung Sommersemester 2021}
	
%\TUBAFTitelseitebilder{\includegraphics{title_page_pic_.jpg}}
%%%%%%%%%%%%%%%%%%%%%%%%%%%%%%%%%%%%%%%%%%%%%%%%%%%%%%%%%%%%%%%%%%%%%%%%%%%%%%%%%%%%%%%%%%%%%%%%%%%%%%%
% pdf-Infos setzen
\hypersetup{%
	pdfauthor={Dominik Kern},			% wird eigentlich von oben übernommen
	pdftitle={Bodendynamik}	% wird eigentlich von oben übernommen
}
%%%%%%%%%%%%%%%%%%%%%%%%%%%%%%%%%%%%%%%%%%%%%%%%%%%%%%%%%%%%%%%%%%%%%%%%%%%%%%%%%%%%%%%%%%%%%%%%%%%%%%%


\begin{document}
\maketitle

%\section{Wellenausbreitung im Untergrund}
\section{Elastisches dreidimensionales Kontinuum}

\begin{frame}
\frametitle{Übersicht}
\begin{center}
\includegraphics[width=0.2\textwidth]{fig_img/youtube.png}   
\end{center}

\href{https://www.youtube.com/watch?v=Za_22xo7ZQQ}{\textsl{IRIS Earthquake Science: 3-component Seismogram --- Capturing the motion of an earthquake.}}

\end{frame}

\begin{frame}
\frametitle{Mechanische Grundlagen}
Bodenverhalten, im einfachsten Fall isotrop, linear elastisches
\begin{align*}
 \sigma = ev + es
\end{align*}
Impulsbilanz damit
\begin{align*}
 \rho \ddot{a} =sigma
\end{align*}


\end{frame}


\subsection{Raumwellen}
% Vrettos2017
% Schmidt2017

\begin{frame}
\frametitle{Video}
\begin{center}
\includegraphics[width=0.2\textwidth]{fig_img/youtube.png}   
\end{center}


\href{https://www.youtube.com/watch?v=gjRGIpP-Qfw}{\textsl{Keith Miller: Demonstrating P and S Seismic Waves}}

\end{frame}


\begin{frame}
\frametitle{Ebene Welle}
in x-Richtung laufende ebene Welle, zwei Lösungen
\end{frame}


\begin{frame}
\frametitle{P-Wellen}
\begin{figure}
\includegraphics[width=\textwidth]{fig_img/p_wave} 
\caption*{\cite{Vrettos2017}}
\end{figure}
wasserabhänging und nu
\end{frame}


\begin{frame}
\frametitle{S-Wellen}
\begin{figure}
\includegraphics[width=\textwidth]{fig_img/s_wave} 
\caption*{\cite{Vrettos2017}}
\end{figure}
wasserunabhängig
\end{frame}


\begin{frame}
\frametitle{Reflektion am Rand}

\end{frame}


\begin{frame}
\frametitle{Reflektion und Transmission am Übergang}
\begin{figure}
\includegraphics[width=0.9\textwidth]{fig_img/wave_transition} 
\caption*{\cite{Vrettos2017}}
\end{figure}

Snell
Totalreflektion
\end{frame}

\begin{frame}
\frametitle{Energie}
\begin{figure}
\includegraphics[width=0.475\textwidth]{fig_img/wave_geometrical_damping} 
\caption*{\cite{Schmidt2017}}
\end{figure}
Energie und Amplitude
\end{frame}




\subsection{Oberflächenwellen}
% Vrettos2017
% Schmidt2017

\begin{frame}
\frametitle{Video}
\begin{center}
\includegraphics[width=0.2\textwidth]{fig_img/youtube.png}   
\end{center}

\href{https://www.youtube.com/watch?v=6yXgfYHAS7c}{\textsl{Wolfram: Propagation of Seismic Waves: Rayleigh waves}}

\href{https://www.youtube.com/watch?v=t7wJu0Kts7w}{\textsl{Wolfram: Propagation of Seismic Waves: Love waves}}

\end{frame}


\begin{frame}
\frametitle{Rayleigh Wellen}
\begin{figure}
\includegraphics[width=\textwidth]{fig_img/rayleigh_wave} 
\caption*{\cite{Vrettos2017}}
\end{figure}

Beschreibung

\end{frame}


\begin{frame}
\frametitle{Rayleigh Wellen}

\begin{figure}
\includegraphics[width=0.8\textwidth]{fig_img/rayleigh_depth} 
\caption*{\cite{Schmidt2017}}
\end{figure}
Dispersion

\end{frame}


\begin{frame}
\frametitle{Kreisfundament}
\begin{figure}
\includegraphics[width=0.9\textwidth]{fig_img/point_load_on_half_space} 
\caption*{\cite{Schmidt2017}}
\end{figure}
Diskussion der Wellen

\end{frame}


\begin{frame}
\frametitle{Love Wellen}
\begin{figure}
\includegraphics[width=\textwidth]{fig_img/love_wave} 
\caption*{\cite{Vrettos2017}}
\end{figure}
Beschreibung
\end{frame}


\begin{frame}
\frametitle{Erdbeben}
\begin{figure}
\includegraphics[width=\textwidth]{fig_img/love_wave_configuration} 
\caption*{\cite{Vrettos2017}}
\end{figure}

siehe 1.5D
\end{frame}




\subsection{Abschließende Bemerkungen}
\begin{frame}
\frametitle{Zahlenwerte}
Fels, Sand
\end{frame}


\begin{frame}
\frametitle{Weitere Einflussfaktoren}
Tiefenabhängigkeit, Grundwasser/TPM,...
Inhomogenitäten
Beugung, Brechung
\end{frame}



%%%%%%%%%%%%%%%%%%%%%%%%%%%%%%%%%%%%%%%%%%%%%%%%

\section*{Literaturverzeichnis}

\begin{frame}[allowframebreaks]{}
	\printbibliography
\end{frame}
\end{document}
